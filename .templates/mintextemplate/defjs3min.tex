%-----------------------------------------------------------------------
%
%                    Style--File fuer Publikationen
%                                des
%                      Instituts fuer Mechanik
%      Fakultaet fuer Ingenieurwissenschaften - Abtl. Bauwissenschaften
%                  Prof. Dr.-Ing. habil. Joerg Schroeder
%
%   Revised 10/03/2010  Joerg Schroeder
%   Revised 10/03/2010  put new symbols for Homogenization, JS
%
% Modifikation:
%   DoBr 07-jun-2010: Abaenderung des Befehls \boldvars
%   DoBr 08-jun-2010: Abaenderung des Befehls \kr
%   DoBr 12-aug-2010: Zusaetzliche {} in der Definition
%                     von \Btime und \Bnull
%   DoBr 06-sep-2010: Korrektur der Befehel \devb und \Devb
%   DoBr 07-jan-2012: Neue Befehle \cdotu, \cdotd, \cdott, \cdotn, \cotx
%   DoBr 29-jul-2013: Kompaiblitaet zu pdflatex (package: pst-all)
%   DoBr 30-jul-2013: Kompaiblitaet zu ppower4-Vortraegen
%   DoBr 16-jan-2014: Package Option ``BoldVarsRoman''
%-----------------------------------------------------------------------
% Documentclass abhaengige Anpassungen
%-----------------------------------------------------------------------
% equations.............................................................
\newcommand {\eb}{\begin{equation}}       % mit Gleichungsnummer
\newcommand {\ee}{\end{equation}}
\newcommand {\ebn}{\begin{displaymath}}   % ohne Gleichungsnummer
\newcommand {\een}{\end{displaymath}}
\newcommand {\eab}{\begin{eqnarray}}      % Array mit Gleichungsnummer
\newcommand {\eae}{\end{eqnarray}}
\newcommand {\eabn}{\begin{eqnarray*}}    % Array ohne Gleichungsnummer
\newcommand {\eaen}{\end{eqnarray*}}
%\newcommand {\fboxeb}[1]{\begin{equation}
%                      \fbox{#1}
%\newcommand {\fboxee}{\end{equation}}
%
% Gleitobjekte
\newcommand{\setfigcapstyle}{\renewcommand{\captionlabelfont}{\footnotesize\bf}
                             \renewcommand{\captionfont}{\footnotesize}
                             \setlength{\captionmargin}{1cm}}
\newcommand{\settabcapstyle}{\renewcommand{\captionlabelfont}{\footnotesize\bf}
                             \renewcommand{\captionfont}{\footnotesize}
                             \setlength{\captionmargin}{1cm}}
\newcommand{\setnormcapstyle}{\renewcommand{\captionlabelfont}{\normalsize\bf}
                             \renewcommand{\captionfont}{\normalsize}
                             \setlength{\captionmargin}{0cm}}

\newenvironment{Figure}[1][0]{\setfigcapstyle \begin{figure}[#1]}
                             {\end{figure} \setnormcapstyle}

\newenvironment{Table}[1][0]{\settabcapstyle \begin{table}[#1]}
                            {\end{table} \setnormcapstyle}

% allgemeine Vereinfachungen.............................................
\newcommand {\af}{\sc}                    % \sc
\newcommand {\ce}{\centerline}            % centerline
\newcommand {\Disp}{\displaystyle}        % Disp
\providecommand {\dfrac}{\Disp\frac}      % Disp und frac
\newcommand {\ul}{\underline}             % underline
%
%-----------------------------------------------------------------------
%   BOLD FACE GREEK:  All In Format: \B***** .
%-----------------------------------------------------------------------
\def\pmb#1{\setbox0=\hbox{#1}%
  \kern-.025em\copy0\kern-\wd0
  \kern.05em\copy0\kern-\wd0
  \kern-.025em\raise.0433em\box0 }
%
\newcommand{\BGamma     }{{\bm{\Gamma      }}}
\newcommand{\BDelta     }{{\bm{\Delta      }}}
\newcommand{\BTheta     }{{\bm{\Theta      }}}
\newcommand{\BLambda    }{{\bm{\Lambda     }}}
\newcommand{\BXi        }{{\bm{\Xi         }}}
\newcommand{\BPi        }{{\bm{\Pi         }}}
\newcommand{\BSigma     }{{\bm{\Sigma      }}}
\newcommand{\BUpsilon   }{{\bm{\Upsilon    }}}
\newcommand{\BPhi       }{{\bm{\Phi        }}}
\newcommand{\BPsi       }{{\bm{\Psi        }}}
\newcommand{\BOmega     }{{\bm{\Omega      }}}
\newcommand{\Balpha     }{{\bm{\alpha      }}}
\newcommand{\Bbeta      }{{\bm{\beta       }}}
\newcommand{\Bgamma     }{{\bm{\gamma      }}}
\newcommand{\Bdelta     }{{\bm{\delta      }}}
\newcommand{\Bepsilon   }{{\bm{\epsilon    }}}
\newcommand{\Bzeta      }{{\bm{\zeta       }}}
\newcommand{\Beta       }{{\bm{\eta        }}}
\newcommand{\Btheta     }{{\bm{\theta      }}}
\newcommand{\Biota      }{{\bm{\iota       }}}
\newcommand{\Bkappa     }{{\bm{\kappa      }}}
\newcommand{\Blambda    }{{\bm{\lambda     }}}
\newcommand{\Bmu        }{{\bm{\mu         }}}
\newcommand{\Bnu        }{{\bm{\nu         }}}
\newcommand{\Bxi        }{{\bm{\xi         }}}
\newcommand{\Bpi        }{{\bm{\pi         }}}
\newcommand{\Brho       }{{\bm{\rho        }}}
\newcommand{\Bsigma     }{{\bm{\sigma      }}}
\newcommand{\Btau       }{{\bm{\tau        }}}
\newcommand{\Bupsilon   }{{\bm{\upsilon    }}}
\newcommand{\Bphi       }{{\bm{\phi        }}}
\newcommand{\Bchi       }{{\bm{\chi        }}}
\newcommand{\Bpsi       }{{\bm{\psi        }}}
\newcommand{\Bomega     }{{\bm{\omega      }}}
\newcommand{\Bvarepsilon}{{\bm{\varepsilon }}}
\newcommand{\Bve        }{{\bm{\varepsilon }}}% doppelt definiert
\newcommand{\Bvartheta  }{{\bm{\vartheta   }}}
\newcommand{\Bvarpi     }{{\bm{\varpi      }}}
\newcommand{\Bvarrho    }{{\bm{\varrho     }}}
\newcommand{\Bvarsigma  }{{\bm{\varsigma   }}}
\newcommand{\Bvarphi    }{{\bm{\varphi     }}}
%
%-----------------------------------------------------------------------
%   GREEK:
%-----------------------------------------------------------------------
%
\def\ve {{\varepsilon}}
\def\Bve {{\Bvarepsilon}}
%
%-----------------------------------------------------------------------
%   BOLD FACE MATH
%-----------------------------------------------------------------------
%
% Boldface variables depending of boolean variable 'boldeqnitalic'
\newcommand{\boldvars}[1]{{\text{\bfseries\rmfamily{#1}}}}% false



\newcommand{\bA}{{\boldvars{A}}}
\newcommand{\bB}{{\boldvars{B}}}
\newcommand{\bC}{{\boldvars{C}}}
\newcommand{\bD}{{\boldvars{D}}}
\newcommand{\bE}{{\boldvars{E}}}
\newcommand{\bF}{{\boldvars{F}}}
\newcommand{\bG}{{\boldvars{G}}}
\newcommand{\bH}{{\boldvars{H}}}
\newcommand{\bI}{{\boldvars{I}}}
\newcommand{\bJ}{{\boldvars{J}}}
\newcommand{\bK}{{\boldvars{K}}}
\newcommand{\bL}{{\boldvars{L}}}
\newcommand{\bM}{{\boldvars{M}}}
\newcommand{\bN}{{\boldvars{N}}}
\newcommand{\bO}{{\boldvars{O}}}
\newcommand{\bP}{{\boldvars{P}}}
\newcommand{\bQ}{{\boldvars{Q}}}
\newcommand{\bR}{{\boldvars{R}}}
\newcommand{\bS}{{\boldvars{S}}}
\newcommand{\bT}{{\boldvars{T}}}
\newcommand{\bU}{{\boldvars{U}}}
\newcommand{\bV}{{\boldvars{V}}}
\newcommand{\bW}{{\boldvars{W}}}
\newcommand{\bX}{{\boldvars{X}}}
\newcommand{\bY}{{\boldvars{Y}}}
\newcommand{\bZ}{{\boldvars{Z}}}
%
\newcommand{\ba}{{\boldvars{a}}}
\newcommand{\bb}{{\boldvars{b}}}
\newcommand{\bc}{{\boldvars{c}}}
\newcommand{\bd}{{\boldvars{d}}}
\newcommand{\be}{{\boldvars{e}}}
\newcommand{\bbf}{{\boldvars{f}}}
\newcommand{\bg}{{\boldvars{g}}}
\newcommand{\bh}{{\boldvars{h}}}
\newcommand{\bi}{{\boldvars{i}}}
\newcommand{\bj}{{\boldvars{j}}}
\newcommand{\bk}{{\boldvars{k}}}
\newcommand{\bl}{{\boldvars{l}}}
\newcommand{\bbm}{{\boldvars{m}}}
\newcommand{\bn}{{\boldvars{n}}}
\newcommand{\bo}{{\boldvars{o}}}
\newcommand{\bp}{{\boldvars{p}}}
\newcommand{\bq}{{\boldvars{q}}}
\newcommand{\br}{{\boldvars{r}}}
\newcommand{\bs}{{\boldvars{s}}}
\newcommand{\bt}{{\boldvars{t}}}
\newcommand{\bu}{{\boldvars{u}}}
\newcommand{\bv}{{\boldvars{v}}}
\newcommand{\bw}{{\boldvars{w}}}
\newcommand{\bx}{{\boldvars{x}}}
\newcommand{\by}{{\boldvars{y}}}
\newcommand{\bz}{{\boldvars{z}}}
%
%-------  mathbb ---------------------------------
\newcommand{\CC}{\mathbb{C}}
\newcommand{\MM}{\mathbb{M}}
\newcommand{\NN}{\mathbb{N}}
\newcommand{\RR}{\mathbb{R}}
%
%-----------------------------------------------------------------------
%   BOLD FACE MATH ITALIC (schraeg), Format: \fit{*} oder \olfit{*}
%-----------------------------------------------------------------------
%
\newcommand{\fit}[1]{{\bm{#1}}}
\newcommand{\olfit}[1]{\overline{\bm{#1}}}
%
%-----------------------------------------------------------------------
%   BOLD FACE MATH Latin (gerade),   Format: \fla{*} oder \olfla{*}
%-----------------------------------------------------------------------
%
\newcommand{\fla}[1]{{\bm{\mathrm{#1}}}}
\newcommand{\olfla}[1]{{\overline{\bm{\mathrm{#1}}}}}
%
%-----------------------------------------------------------------------
%   Mathematische Akzente:  All In Format: \st{*}
%-----------------------------------------------------------------------
%
\newcommand{\ostar}[1]{\overset{\star}{#1}}           % star^{.....}
\newcommand{\oast}[1]{\overset{\ast}{#1}}             % ast^{.....}
\newcommand{\otriangle}[1]{\overset{\triangle}{#1}}   % triangle^{.....}
\newcommand{\oDelta}[1]{\overset{\Delta}{#1}}         % Delta^{.....}
\newcommand{\ov}[2]{\overset{#1}{#2}}                 % {,,,}^{.....}
%
%=======================================================================
%   M A T H   S Y M B O L S   ( C O N T I )
%=======================================================================
%
% partial derivatives......................................................
\newcommand{\p}[1] { \partial_{#1} }                  %dervative by tensor
\newcommand{\pp}[2]{ \frac{\partial{#1}} {\partial{#2}} }
\newcommand{\ppp}[2]{ \frac{\partial^2{#1}} {\partial{#2}} }
\newcommand{\pppp}[3]{ \frac{\partial^2{#1}} {\partial{#2} \partial{#3}} }
% total derivatives
\newcommand{\dd}[2]{ \frac{d{#1}}        {d{#2}}        }  %Leibniz derivative
\newcommand{\ddt}{{\pp{}{t}}}                              %\partial/\partial t
%
% mappings..............................................................
\newcommand{\map}[3]{ {#1}:{#2} \to {#3} }       %map, domains
\newcommand{\maps}[3]{ {#1}:{#2} \mapsto {#3} }  %map, images
\newcommand{\phit}{\Bvarphi_t}                   %\phi_t
%
% Ueberschiebungen
\DeclareMathOperator{\cdotu}{\cdot}
\DeclareMathOperator{\cdotd}{:}
\DeclareMathOperator{\cdott}{\therefore}
\newcommand{\cdotn}{\stackrel{(n)}{\cdotu}}
\newcommand{\cdotx}[1]{\stackrel{(#1)}{\cdotu}}
%
% topology..............................................................
\newcommand{\B}{{\cal B}}                         %Referenzkonfiguration
\newcommand{\Bt}{{\cal S}}                        %Momentankonfiguration
\newcommand{\Bnull}{{{\cal B}_0}}                 %Referenzkonfiguration
\newcommand{\Btime}{{{\cal B}_t}}                 %Momentankonfiguration
\newcommand{\OG}{{\cal O}}
\newcommand{\SOG}{{\cal SO}}
\newcommand{\MO}{{\cal MO}}
\newcommand{\F}{{\cal F}}
\newcommand{\RVE}{{\cal V}}     %Referenzkonfiguration der Mikrostruktur
%
\newcommand{\jumpl}{{[\kern-.15em[}}
\newcommand{\jumpr}{{]\kern-.15em]}}
%
% Lie-derivatives........................................................
\newcommand{\Lieder}{\mbox{\boldmath${\pounds}$}}%symbol for LIE derivative
\newcommand{\Lied}{{\Lieder_{\triangle}}}       %L_triangle
\newcommand{\Liev}{{\Lieder_{\bv}}}             %L_v
\newcommand{\Lvg}{{\Liev \bg}}                  %L_v g
\newcommand{\tiLiev}{{\tilde{\Lieder}_{\bv} }}  %\tilde L_v
\newcommand{\tiLvg}{{\tilde{\Lieder}_{\bv} \bg}}%\tilde L_v g
\newcommand{\barLiev}{{\bar{\Lieder}_{\bv} }}   %\bar L_v
\newcommand{\Lievp}{{\Lieder_{\bv}^p}}          %L_v^p
\newcommand{\Liedp}{{\Lieder_{\bv}^p}}          %L_v^p
\newcommand{\Lvtau}{{\Liev \Btau}}              %L_v tau
\newcommand{\Lvb}{{\Liev \bbb}}                 %L_v b
\newcommand{\Lvbe}{{\Liev \bbb^e}}              %L_v b^e
\newcommand{\Lvpce}{{\Lievp \bC^e}}             %L_v C^e
\newcommand{\Lvpse}{{\Lievp \bS^e}}             %L_v^p S^e
\newcommand{\Lvpseb}{{\Lievp \bar\bS}}          %L_v^p \barS
%
% Tangent map............................................................
\def\Ft{{\bF_t}}                % tangent map at time t
\def\phitX{{\phit(\bX)}}            % point x: image of X at t
\def\phitB{{\phit(\B)}}             % current configuration S: image of B at t
\def\pull{{\it 'pull-\-back' }}     % Pull-Back
\def\push{{\it 'push-\-forward' }}  % Push-Forward
%
% deformationtensors.....................................................
\def\Cbar{{\bar{\bC}_t}}           % elastic right Cauchy Green
\def\dotCbar{\dot{\bar{\bC}}_t}    % elastic right Cauchy Green, rate
\def\LvpCbar{{\Lievp \bar{\bC}_t}} % elastic right Cauchy Green, plastic Lie
% timederivatives........................................................
\def\Lpbar{\bar{\bL}_t^p}          % plastic velocity gradient
\def\Lpbart{\bar{\bL}_t^{pT}}      % plastic velocity gradient, transposed
\def\Dpbar{\bar{\bD}_t}            % plastic rate of deformation
\def\Wpbar{\bar{\bW}_t}            % plastic spin
%
% Tangent--Spaces........................................................
\def\TBX{{T_{X}\B}}               % tangent space of B at X
\def\CBX{{T_{X}^{\ast}\B}}        % cotangent space of B at X
\def\TIN{{\bar{T}_{X}\B}}         % vector space at X,intermediate config.
\def\CIN{{\bar{T}_{X}^{\ast}\B}}  % dual vector space at X,intermediate config.
\def\TSx{{T_{x}\phitB}}           % tangent space of S at x
\def\CSx{{T_{x}^{\ast}\phitB}}    % cotangent space of S at x
%
% Neighborhoods..........................................................
\def\NX{{\cal N}_{X}}         % neighborhood of X
\def\INT{\bar{\cal N}_{X}}    % intermediate configuration
\def\Nx{{\cal N}_{x}}         % neighborhood of x
%
% nablas.................................................................
\def\nablaX#1{{\nabla}_{X}{#1}}     % Grad, material gradient
\def\nablax#1{{\nabla}_{x}{#1}}     % grad, spatial gradient
%
% Begin: Definitionen von P. Neff
%-------  Raeume, etc ---------------------------
\newcommand{\Lp}[1]{L^{#1}(\Omega)}
\newcommand{\Wmp}[2]{W^{#1,#2}(\Omega)}
\newcommand{\Hmp}[2]{H^{#1,#2}(\Omega)}
\newcommand{\Hmpo}[2]{H_{\circ}^{#1,#2}(\Omega)}
\newcommand{\Co}{C_0^{\infty}(\Omega)}
\newcommand{\Cgam}{C^\infty(\Omega,\Gamma)}
%\newcommand{\qed}{\hfill $\blacksquare$\\}
\newcommand{\qeda}{\tag*{$\blacksquare$} }
\newcommand{\Psym}{P\mathbb{S}ym}
%
%-------  Normen ---------------------------------
\newcommand{\abs}[1]{|#1|}
\newcommand{\norm}[1]{\|#1\|}
\newcommand{\HmpOnorm}[3]{{\|#1\|}_{#2,#3,\Omega}}
\newcommand{\LpOnorm}[2]{{\|#1\|}_{#2,\Omega}}
\newcommand{\Skalint}[2]{{\langle #1,#2\rangle}_{\Omega}}
\newcommand{\Skalprod}[2]{\langle #1,#2\rangle}
\newcommand{\rSkalprod}[2]{ {\langle #1,#2\rangle}^{\bullet} }
\newcommand{\Mprod}[2]{ {\langle #1 ,#2\rangle} }
\newcommand{\MSkalprod}[2]{ {\langle #1,#2\rangle}_{\MM^{3\times3}} }
\newcommand{\Rprod}[2]{ {\langle #1,#2\rangle}_{\RR^{3}} }
\newcommand{\RSkalprod}[2]{ {\langle #1,\!#2\rangle}_{\RR^{3}} }
\newcommand{\rLpOnorm}[2]{ {{\|#1\|}_{#2,\Omega}^{\bullet}} }
\newcommand{\rHnorm}[1]{ {{\rule[-0.1cm]{0.025cm}{0.45cm}#1
                      \rule[-0.1cm]{0.025cm}{0.45cm} }_{\Omega}^{\bullet}} }
\newcommand{\Mnorm}[1]{ {{\|#1\|}_{\MM^{3\times3}} } }
\newcommand{\Rnorm}[1]{ {{\|#1\|}_{\RR^3}} }
\newcommand{\Mnnorm}[2]{ {{\|#1\|}_{#2,\MM^{3\times3}} }}
% End: Definitionen von P. Neff
%
% special operators.....................................................
%
%             Font fuer Assembly-Operator
%
%%%%%%%%%%%%%%%%%%%%%%%%%%%%%%%%%%%%%%%%%%%%%%%%%%%%%%%%%%%%%%%%%%%%
%%%                           m_H
%%%%   Definition von          A      ------>   \Assem
%%%                           e=1
%%%%%%%%%%%%%%%%%%%%%%%%%%%%%%%%%%%%%%%%%%%%%%%%%%%%%%%%%%%%%%%%%%%%
\newfont{\Sf}{cmssbx10 scaled 2074}
%% Assembly Operator
%\newcommand {\Assem}{\unitlength=1mm\begin{picture}(8,7)\put(0,-3){
%       $\stackrel{\stackrel{\mbox{\tiny nele}}{\mbox{\Sf A}}}{
%\mbox{\tiny $e=1$}}$}\end{picture}}
%% Assembly Operator AS
\newcommand {\Assem}{\unitlength=1mm\begin{picture}(8,7)\put(0,-4){
       $\stackrel{\stackrel{\mbox{\tiny $num_{ele}$}}{\mbox{\Sf
A}}}{\mbox{\tiny e = 1}}$}\end{picture} \; }
%% Assembly Operator ohne elemente JS
\newcommand {\Assemjs}{\unitlength=1mm\begin{picture}(8,7)\put(0,-4){
       $\stackrel{\stackrel{\mbox{\small \phantom{$num_{ele}$}}}{\mbox{\Sf
A}}}{\mbox{\small e }}$}\end{picture}}
%% Assembly Operator mit benutzdefinierten Eintraegen
\newcommand{\Assemarg}[2]{{%
  \underset{\mathsmaller{#1}}{\overset{\mathsmaller{#2}}%
  {\raisebox{-0.5ex}{\sffamily\bfseries\larger[3]{A}}}}\;%
  }}
%
%%%%%%%%%%%%%%%%%%%%%%%%%%%%%%%%%%%%%%%%%%%%%%%%%%%%%%%%%%%%%%%%%%%%
%%%
%%%%   Definition von          A      ------>   \Assemo
%%%
%%%%%%%%%%%%%%%%%%%%%%%%%%%%%%%%%%%%%%%%%%%%%%%%%%%%%%%%%%%%%%%%%%%%
%% Assembly Operator ohne Elemente
%%\newcommand {\Assemo}{\unitlength=1mm\begin{picture}(8,7)\put(0,-3){
%%       $\stackrel{\stackrel{\mbox{\tiny }}{\mbox{\Sf A}}}{\mbox{\tiny
%%}}$}\end{picture}}
%% Assembly Operator ohne Elemente
\newcommand {\Assemo}{\unitlength=1mm\begin{picture}(8,7)\put(0,-4){
       $\stackrel{\stackrel{\mbox{\small }}{\mbox{\Sf A}}}{\mbox{\small
}}$}\end{picture} \; }
%
%%%%%%%%%%%%%%%%%%%%%%%%%%%%%%%%%%%%%%%%%%%%%%%%%%%%%%%%%%%%%%%%%%%%
\def\bone{{\bf 1}}           % bold 1
\def\bzero{{\bf 0}}          % bold 0

% Mathematische Operatoren
% 'b'-Kommandos erzeugen 'boxed'-Argumente (eckige Klammern)
\def\adj{\operatorname{adj}}      % adj  ohne Argument
\def\Adj{\operatorname{Adj}}      % Adj  ohne Argument
\def\adjb#1{\adj[{#1}]}           % adjugate boxed Argument Adj[#]
\def\Adjb#1{\Adj[{#1}]}           % Adjugate boxed Argument Adj[#]
\def\Cof{\operatorname{Cof}}      % Cof  ohne Argument
\def\Cofb#1{\Cof[{#1}]}           % Cofactor boxed Argument Cof[#]
\def\det{\operatorname{det}}      % det  ohne Argument
\def\Det{\operatorname{Det}}      % Det  ohne Argument
\def\detb#1{\det[{#1}]}           % Determinat boxed Argument det[#]
\def\Detb#1{\Det[{#1}]}           % Determinat boxed Argument Det[#]
\def\dev{\operatorname{dev}}      % dev ohne Argument
\def\Dev{\operatorname{Dev}}      % Dev ohne Argument
\def\devb#1{\dev[{#1}]}           % Deviator boxed Argument dev[#]
\def\Devb#1{\Dev[{#1}]}           % Deviator boxed Argument Dev[#]
\def\div{\operatorname{div}}      % div ohne Argument
\def\Div{\operatorname{Div}}      % Div ohne Argument
\def\divb#1{\div[{#1}]}           % Divergence boxed Argument div[#]
\def\Divb#1{\Div[{#1}]}           % Divergence boxed Argument Div[#]
\def\grad{\operatorname{grad}}    % grad ohne Argument
\def\Grad{\operatorname{Grad}}    % Grad ohne Argument
\def\gradb#1{\grad[{#1}]}         % gradient boxed Argument grad[#]
\def\Gradb#1{\Grad[{#1}]}         % Gradient boxed Argument Grad[#]
\def\rot{\operatorname{rot}}      % rot ohne Argument
\def\Rot{\operatorname{Rot}}      % Rot ohne Argument
\def\rotb#1{\rot[{#1}]}           % rot boxed Argument rot[#]
\def\Rotb#1{\Rot[{#1}]}           % Rot boxed Argument Rot[#]
\def\skew{\operatorname{skew}}    % skew ohne Argument
\def\Skew{\operatorname{Skew}}    % Skew ohne Argument
\def\skewb#1{\skew[{#1}]}         % skew boxed Argument skew[#]
\def\Skewb#1{\Skew[{#1}]}         % Skew boxed Argument Skew[#]
\def\sym{\operatorname{sym}}      % sym ohne Argument
\def\Sym{\operatorname{Sym}}      % Sym ohne Argument
\def\symb#1{\sym[{#1}]}           % sym boxed Argument sym[#]
\def\Symb#1{\Sym[{#1}]}           % Sym boxed Argument Sym[#]
\def\tr{\operatorname{tr}}        % tr ohne Argument
\def\trb#1{\tr[{#1}]}             % tr boxed Argument tr[#]
%
\def\SL3{{\itseries SL(3)}}       % SL(3)
\def\SO3{{\it SO(3)}}             % SO(3)
%
\def\exp{\operatorname{exp}}      % exp
\def\ln{\operatorname{ln}}        % ln
%
\def\ndim{n_{\rm dim}}            % n_dim
\def\trial{\operatorname{trial}}  % trial
\def\rve{${\cal RVE}$}      % RVE
%
% fracs................................................................
\def\half{{\textstyle{{\frac{1}{2}}}}}         % 1/2
\def\third{{\textstyle{1 \over 3}}}            % 1/3
\def\fourth{{\textstyle{{1 \over 4}}}}         % 1/4
\def\sixth{{\textstyle{{1 \over 6}}}}          % 1/6
\def\eigth{{\textstyle{{1 \over 8}}}}          % 1/6
\def\twothird{{\textstyle {{2 \over 3}}}}      % 2/3
\def\thirdtwo{{\textstyle {{3 \over 2}}}}      % 3/2
\def\sqtwo3{{\textstyle {\sqrt{2 \over 3}}}}   % sqrt(2/3)
%
% special tensors......................................................
\newcommand{\reac}{{\rm reac}}
\newcommand{\pind}{{\rm pind}}
\newcommand{\IB}{{\rm I\kern-.25em B}}           % B-Matrix
\newcommand{\ID}{{\rm I\kern-.25em D}}           % enhanced c. of element \bE
\newcommand{\E}{{\rm I\kern-.25em E}}            % enhanced c. of element \bE
\newcommand{\IE}{{\rm I\kern-.25em E}}           % enhanced c. of element \bE
\newcommand{\IF}{{\rm I\kern-.25em F}}           % enhanced c. of element \bF_0
\newcommand{\IH}{{\rm I\kern-.25em H}}           % H
\newcommand{\II}{{\rm I\kern-.18em I}}           % 4th order identitytensor
\newcommand{\ident}{{\rm I\kern-.20em I}}        % Blackboard I
\newcommand{\IK}{{\rm I\kern-.25em K}}           % Lokalisierungstensor
\newcommand{\IL}{{\rm I\kern-.25em L}}           % Lokalisierungstensor
\newcommand{\IP}{{\rm I\kern-.18em P}}           % 4th order projectiontensor
%\newcommand{\IP}{{\rm I\kern-.25em P}}          % 4th order projectiontensor
\def\IR{{\rm I\kern-.25em R}}
\def\IN{{\rm I\kern-.25em N}}
%
\newcommand{\ia}{{\rm\kern.24em                  %fourth order tensor a
   \vrule width.02em height0.9ex depth-.05ex
   \kern-.26em a}}
\newcommand{\ic}{{\rm\kern.24em                  %fourth order tensor c
   \vrule width.02em height0.9ex depth-.05ex
   \kern-.26em c}}
\newcommand{\IA}{{\rm\kern.22em                  %fourth order tensor A
    \vrule width.02em
        height0.5ex depth 0ex
    \kern-.24em A}}
\newcommand{\IC}{{\rm\kern.24em                  %fourth order tensor C
   \vrule width.02em height1.4ex depth-.05ex
   \kern-.26em C}}
\newcommand{\IS}{{\rm\kern.24em                  %fourth order Eshelby tensor
   \vrule width.02em height1.4ex depth-.05ex
   \kern-.26em S}}
%
% verallgemeinerte Kroneckersymbole........................................
\newcommand{\deltaII}[4] { \delta^{#1}{}_{#2}{}^{#3}{}_{#4}   }
\newcommand{\deltaIII}[6] {\delta^{#1}{}_{#2}{}^{#3}{}_{#4}{}^{#5}{}_{#6}   }
%
\newcommand{\veci}[1]{\mbox{$#1_1,\ldots,#1_n$}}
%
% averaging tensors....................................................
\newcommand{\Bsigmaa}[1] { \langle \Bsigma{#1} \rangle   }
\newcommand{\Bvarepsilona}[1] { \langle \Bvarepsilon{#1} \rangle   }
\newcommand{\sigmaa}[2] { \langle \sigma_{#1}{}_{#2} \rangle   }
\newcommand{\varepsilona}[2] { \langle \varepsilon_{#1}{}_{#2} \rangle   }
\newcommand{\ICa}[4] { \langle \ICa_{#1}{}_{#2}_{#3}{}_{#4} \rangle   }
%
%---------------------------------------------------------------------------
%Folienstyle
%---------------------------------------------------------------------------
%RAHMEN
\newcommand{\verbrahmen}[4]
          {\unitlength1cm
           \begin{picture}(18,00)
             \put(10,0.1){\makebox(8,0.5)[r]{#2}}
             \put(0,0.2){#3}
             \put(0,-26.2){{\tiny #1}}
             \put(0,-26){\framebox(18,26)[t]{
                         \begin{minipage}[t]{160mm}
                           \vspace*{2cm}
                           {#4}
                         \end{minipage}

}}
        %\put(0,0){0/0}
        %\put(10,10){10/10}
            \end{picture}
}
%RAHMEN
\newcommand{\rahmen}[4]
          {\unitlength1cm
           \begin{picture}(18,26)
             \put(10,26.1){\makebox(8,0.5)[r]{#2}}
             \put(0,26.2){#3}
             \put(0,-0.3){{\tiny #1}}
             \put(0,0){\framebox(18,26)[t]{
                         \begin{minipage}[t]{160mm}
                           \vspace*{2cm}
                           {#4}
                         \end{minipage}

}}
            \end{picture}
}
%QUERRAHMEN
\newcommand{\querrahmen}[4]
          {\unitlength1cm
           \special{landscape}
           \begin{picture}(26,18)
             \put(18,17.6){\makebox(8.5,0.5)[r]{#2}}
             \put(0,17.7){#3}
             \put(0,-0.3){{\tiny #1}}
             \put(0,0){\framebox(26.5,17.5)[t]{
                         \begin{minipage}[t]{230mm}
                           \vspace*{2cm}
                           {#4}
                         \end{minipage}

}}
            \end{picture}
}
\newcommand{\copyr}{{$\copyright$
                    Prof.~Dr.-Ing.~J\"org Schr\"oder,
                    Institute of Mechanics,
                    Civil Engineering}}
%----------------------------------------------------------------------
% Aenderungen von J.Loeblein
\newcommand{\xfh}{^}
\newcommand{\fett}[1]{%
         \mbox{\boldmath$ #1 $}}
\newcommand{\kr}[1]{%
         $\pscirclebox[framesep=1.5pt,linewidth=0.2pt]{#1}$}
%     $\begin{array}{c}\bigcirc\\[-5.3mm]{\scriptstyle #1}
%      \end{array}$}







